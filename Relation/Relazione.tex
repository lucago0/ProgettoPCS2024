\documentclass[a4paper]{article}
\usepackage[T1]{fontenc}
\usepackage[utf8]{inputenc}
\usepackage[italian]{babel}
\usepackage{amsthm}
\usepackage{amssymb}
\usepackage{mathtools}
\usepackage{enumerate}
\newcommand{\B}{\mathbf}
\usepackage{graphicx}
\swapnumbers

\title{Progetto Programmazione e Calcolo Scientifico 2024}
\author{Luca Agostino, Tommaso Basile, Andrea Cigna}

\begin{document}

\maketitle

\tableofcontents

\section{Piano passante per una frattura}
Date $N$ fratture con $n_i$ vertici ciascuna, denominiamo con $\B{P}^{(i)}_j$ il vertice $j$ della frattura $i$. I range di valori sono, quindi, $i=0,\dots,N-1$ e $j=0,\dots,n_i-1$. Pertanto, per ogni frattura $i$, abbiamo una $n_i$-upla $(\B{P}^{(i)}_0,\dots,\B{P}^{(i)}_{n_i-1})$. Cerchiamo l'equazione cartesiana del piano $\pi$ contenente la frattura, dando per assunto che effettivamente i vertici forniti appartengono tutti ad uno stesso piano e che, quindi, corrispondano effettivamente a un poligono nello spazio tridimensionale. Per semplicità di notazione omettiamo nelle trattazioni successive l'apice $(i)$, sottointendendo che ci stiamo riferendo alla traccia $i$-esima. Vogliamo arrivare a una scrittura di $\pi$ del tipo $$ax+by+cz=d.$$
Se denotiamo con ${\B{A}\B{B}}=(x_B-x_A,y_B-y_A,z_B-z_A)$ il vettore congiungente $\B{A}$ a $\B{B}$, un generico punto $\B{P}\in\mathbb{R}^3$ appartiene al piano passante per i punti $\B{P}_0,\B{P}_1,\B{P}_2$ se vale che 
\begin{equation}\label{eq:piano1}
(\B{P}_0\B{P}_1\wedge \B{P}_0\B{P}_2)\cdot \B{P}_0\B{P}=0,
\end{equation} 
cioè il vettore che congiunge $\B{P}_0$ a $\B{P}$ deve essere ortogonale al vettore normale al piano (detto giacitura) $\B{P}_0\B{P}_1\wedge \B{P}_0\B{P}_2$. 
La \eqref{eq:piano1} equivale a porre nullo il determinante della matrice avente per righe le componenti dei $3$ vettori rispetto alla base canonica di $\mathbb{R}^3$: 
$$
\det\begin{pmatrix} x-x_0 & y-y_0 & z-z_0 \\ x_1-x_0 & y_1-y_0 & z_1-z_0 \\x_2-x_0 & y_2-y_0 & z_2-z_0 \end{pmatrix}=0.
$$
Sviluppando il determinante con la regola di Laplace rispetto alla prima riga otteniamo un'equazione del tipo 
\begin{equation}\label{eq:piano2}
a\left(x-x_0\right)+b\left(y-y_0\right)+c\left(z-z_0\right)=0,
\end{equation} 
dove 
\[
\begin{aligned}
& a=\operatorname{det}\left(\begin{array}{ll} y_1-y_0 & z_1-z_0 \\ y_2-y_0 & z_2-z_0 \end{array}\right), \\
& b=-\operatorname{det}\left(\begin{array}{ll}x_1-x_0& z_1-z_0 \\ x_2-x_0 & z_2-z_0 \end{array}\right),\\
& c=\operatorname{det}\left(\begin{array}{ll} x_1-x_0 & y_1-y_0 \\ x_2-x_0 & y_2-y_0 \end{array}\right). 
\end{aligned}
\]
Esplicitiamo i prodotti a primo membro dell'equazione \eqref{eq:piano2} 
\[
\begin{aligned}
& a\left(x-x_0\right)+b\left(y-y_0\right)+c\left(z-z_0\right)=0, \\
& a x-a x_0+b y-b y_0+c z-c z_0=0, \\
& a x+b y+c z-a x_0-b y_0-c z_0=0.
\end{aligned}
\]
Chiamiamo
$$
d=-a x_0-b y_0-c z_0.
$$ 
Così facendo abbiamo ricavato l'equazione cartesiana del piano cercato 
$$
ax+by+cz+d=0,
$$ 
con $a,b,c$ non tutti nulli, non essendo tutti e tre i punti allineati.

\section{Retta di intersezione tra piani}
Date due fratture diverei $i$ e $j$, grazie a quanto detto sopra, riusciamo a trovare l'equazione cartesiana dei due piani contententi le fratture. Nel caso in cui i due piani non siano paralleli, ci proponiamo di ricavare l'equazione in forma parametrica della retta $r$ che si genera dall'intersezione dei due piani; siamo, cioè, in cerca di una scrittura del tipo $$\B{OP}=\B{OQ}+t\B{v},$$ con $t$ parametro reale e $\B{v}$ la direzione della retta. Scriviamo in un sistema le equazioni per i due piani in questo modo: 
$$
\begin{cases}
a_1x+b_1y+c_1z=d_1\\ 
a_2x+b_2y+c_2z=d_2
\end{cases}
$$ 
dove i pedici indicano il piano rappresentato. Per i due piani definiamo con 
$$
\B{n}_1=\B{P}_0\B{P}_1\wedge \B{P}_0\B{P}_2,\quad \B{n}_2 = \B{P}'_0\B{P}'_1\wedge \B{P}'_0\B{P}'_2
$$ 
le due giaciture. La direzione individuata dalla retta di intersezione è data dal vettore normale a entrambe le giaciture dei piani: 
$$
\B{v}=\B{n}_1\wedge \B{n}_2.
$$ 
Al fine di trovare $\B{Q}$, se $c_1,c_2\neq 0$, battezziamo $z$ come parametro libero e, ponendolo pari a $0$, risolviamo il sistema lineare 
$$
\begin{cases}
a_1x+b_1y=d_1\\ 
a_2x+b_2y=d_2
\end{cases}
$$ 
trovando $\B{OQ}=(\overline{x},\overline{y},0)$. Allora la retta di intersezione $r$ è data in forma parametrica da 
$$
\begin{cases}
x=\overline{x}+tv_x \\ 
y=\overline{y}+tv_y \\ 
z=tv_z.
\end{cases}
$$

\section{Punti di intersezione}
Per ogni coppia di vertici consecutivi $\B{P}_j,\B{P}_{j+1}$ scriviamo la retta passante per entrambi $r_j$ attraverso l'ascissa curvilinea $s$ in questo modo: 
$$
\B{OP}=\B{OP}_j+s\B{P}_j\B{P}_{j+1},
$$
pertanto solo per valori di $s$ in $(0,1)$ otteniamo un punto del lato della frattura. 
Consideriamo adesso due fratture, attraverso i punti $1$ e $2$ troviamo i piani che li contengono e la retta di intersezione (se esiste) tra le due.

Per ogni coppia di vertici consecutivi della prima frattura intersechiamo la retta $r$ di intersezione e la retta $r_j$:
$$
\begin{cases}
\B{OP}=\B{OQ}+t\B{v} \\ 
\B{OP}=\B{OP}_j+s\B{P}_j\B{P}_{j+1}
\end{cases}
$$
per cui abbiamo $$\B{OQ}+t\B{v}=\B{OP}_j+s\B{P}_j\B{P}_{j+1},$$ ossia un sistema di $3$ equazioni nelle incognite $t,s$. Riscriviamolo in tal modo:
\begin{equation} \label{sis:parametri}
\begin{cases}
v_xt-(x_{j+1}-x_j)s=x_j-\overline{x} \\
v_yt-(y_{j+1}-y_j)s=y_j-\overline{y} \\
v_zt-(z_{j+1}-z_j)s=z_j-\overline{z} \\
\end{cases}
\end{equation}
Potrebbe succedere che il sistema non ammetta soluzioni, in tal caso le due rette non si intersecano, cioè la traccia non attraversa il lato $\B{P}_j\B{P}_{j+1}$. Altrimenti, in caso di sistema determinato otterremo la coppia $(s,t)$. Se $s\in[0,1]$ otteniamo effettivamente che il punto di intersezione giace sul lato $\B{P}_j\B{P}_{j+1}$ della prima frattura. Iteriamo il processo per tutti gli $n_i$ vertici della frattura e troveremo due punti $\B{Q}_1$ e $\B{Q}_2$, trovati attraverso l'intersezione tra $r$ e due lati distinti della frattura.

Alla stessa maniera, effettuiamo questi calcoli per la seconda frattura, trovando o intersezione nulla per ogni lato o due altri due punti $\B{Q}_3$ e $\B{Q}_4$.

Nel caso in cui troviamo i 4 punti di intersezione, associamo ad ognuno il valore di $t$ corrispondente trovato nel sistema \eqref{sis:parametri} e costruiamo due intervalli dell'asse reale ordinando i valori dei parametri: otteniamo allora $(a,b)$ e $(c,d)$, dove $a,b\in\{t_1,t_2\}$ e $c,d\in\{t_3,t_4\}$. Se poniamo $a=\text{min}\{t_1,t_2\}$, $b=\text{max}\{t_1,t_2\}$, $c=\text{min}\{t_3,t_4\}$ e $d=\text{max}\{t_3,t_4\}$, attraverso una funzione che calcoli l'intersezione tra questi due intervalli, verifichiamo che
\begin{enumerate} [(i)]
\item se $(a,b)\cap(c,d)=(a,c)$, allora si tratta di una traccia non passante e i suoi vertici sono $\B{Q}_1$ e $\B{Q}_3$;
\item se $(a,b)\cap(c,d)=(a,b)$, allora si tratta di una traccia passante per il poligono $1$ e i suoi vertici sono $\B{Q}_1$ e $\B{Q}_2$;
\item se $(a,b)\cap(c,d)=(b,d)$, allora si tratta di una traccia non passante e i suoi vertici sono $\B{Q}_2$ e $\B{Q}_4$;
\item se $(a,b)\cap(c,d)=(c,d)$, allora si tratta di una traccia passante per il poligono $2$ e i suoi vertici sono $\B{Q}_3$ e $\B{Q}_4$.
\end{enumerate}
Attraverso questo algoritmo abbiamo trovato i vertici delle tracce, il loro valore di \textbf{Tips} e anche le intersezioni dei prolungamenti di esse sui poligoni, il che risulterà molto utile in seguito.

\section{Determinazione dei sotto-poligoni generati per ogni frattura}
Per ogni frattura $j=1,\dots,N$, consideriamo inizialmente tutte le tracce passanti per lo stesso e, attraverso i valori di $t$ trovati nelle sezioni precedenti (ossia i valori scartati dopo l'intersezione dei due intervalli), identifichiamo i vertici dei nuovi sotto-poligoni sempre allo stesso modo, ossia:$$\B{OP}=\B{OQ}+t^*\B{v},$$ dove $t^*$ è il valore del parametro reale corrispondente a un punto che giace su un lato della frattura e sulla retta contenente la traccia. Per ordinare i punti in senso antiorario dobbiamo ricordare che l'intersezione tra la retta $r_j$ e la retta $r$ è avvenuta prendendo i vertici $\B{P}_j$ e $\B{P}_{j+1}$, che sono già ordinati in senso antiorario. Nel momento in cui $r$ ed $r_j$ risultano in un'intersezione per cui effettivamente il punto $\B{Q}$ giace sul lato (e, quindi, il valore associato $s\in[0,1]$), scartiamo tutti i vertici $\B{P}_{j+1},\dots,\B{P}_{j+k}$, dove $\B{P}_{j+k}$ è il vertice del lato $\B{P}_{j+k}\B{P}_{j+k+1}$ per cui avviene che l'altra intersezione con la retta $r$ è ammissibile. Il nuovo sotto-poligono generato avrà gli stessi vertici della frattura originale, esclusi i vertici $\B{P}_{j+1},\dots,\B{P}_{j+k}$, che verranno sostituiti dai due punti della retta contenente la traccia giacenti sui lati del poligono. 
\\ [2mm]
L'idea alla base della determinazione dei sotto-poligoni è la seguente: analizzando una frattura alla volta, esaminiamo la prima traccia per ordine di lunghezza e valore di Tips, dividiamo il poligono originale in due sotto-poligoni e passiamo alla seconda traccia. A questo punto si possono presentare due casi:
\begin{enumerate} [(i)]
\item  se la traccia è contenuta in uno solo dei sotto-poligoni, allora procederemo al taglio del sotto-poligono in questione;;
\item altrimenti entrambi i sotto-poligoni dovranno essere divisi ulteriormente;
\end{enumerate}

In generale, focalizzandosi su una traccia andremo a tagliare il sotto-poligono che la contiene finché uno dei suoi estremi interseca un lato del sotto-poligono stesso. Non appena l'intersezione della traccia coi lati del poligono non è più contenuta nella traccia stessa, sarà operata l'ultima suddivisione prima di passare alla traccia successiva.\\

L'approccio che abbiamo utilizzato per implementare la determinazione dei sotto-poligoni è iterativo e può essere sintetizzato come segue.  \\ [2mm] Per ogni frattura $j=1,\dots,N$, consideriamo la prima traccia che le appartiene che avrà la coppia di vertici $\{\B{T_{1}},\B{T_{2}}\}$, quindi selezioniamo la prima intersezione che avviene tra la retta contente la traccia ed uno dei lati della frattura individuato dalla coppia $\{\B{P_{j}},\B{P_{j+1}}\}$, questo sarà il punto iniziale della suddivisione dei sotto-poligoni. Svolgiamo il sistema per determinare l'intersezione tra la traccia e un lato.

\begin{equation}
\begin{cases}
\B{OP}=\B{OT}_1+s\B{T}_1\B{T}_{2} \\ 
\B{OP}=\B{OP}_j+t\B{P}_j\B{P}_{j+1}
\end{cases}
\label{eq:intersezioni}
\end{equation}

Potrebbe succedere che il sistema non abbia soluzione, in tal caso le rette sono parallele. Altrimenti in caso di sistema determinato otterremo la coppia $( s,t )$. Se $t \in [0,1]$ otteniamo effettivamente che il punto di intersezione giace sul lato, inoltre se $s \in [0,1]$ allora il punto si troverà sulla traccia stessa mentre se non è compreso in questo intervallo sceglieremo tra tutti i valori di $s$ ottenuti quello più vicino all'intervallo $[0,1]$. \\

Adesso dividiamo il lato che contiene il punto di intersezione in due sotto-lati, questi verranno memorizzati con codice identificativo e vertici nella struttura $Polygonal \ mesh$. Inoltre, ad ogni lato creato verranno assegnati: 
\begin{enumerate} 
\item Un booleano per indicare se è attivo ($true$), cioè fa parte della struttura corrente, oppure se non lo è ($false$);
\item I codici identificativi delle fratture adiacenti al lato, al massimo due.
\end{enumerate}

Quindi il lato diviso assumerà il booleano $false$ poiché esce dalla struttura corrente, mentre i due sotto-lati creati acquisiscono attivazione e l'adiacenza della frattura che stiamo analizzando. \\[2mm]
A questo punto è necessario creare due nuove celle 2D, che rappresenteranno i due sotto-poligoni, e assegnare alle celle i rispettivi lati creati. 

Per assegnare i lati del poligono originale nelle corrette celle 2D create, è opportuno memorizzare dapprima il codice identificativo del lato successivo al lato tagliato, che chiameremo $idInitialPoint$,  in questo modo sarà possibile percorrere in senso antiorario i lati del poligono originale a partire da $idInitialPoint$. \\ [1.5mm]
Quindi un nuovo lato sarà inserito: 
\begin{enumerate} 
\item Nel primo sotto-poligono se il lato non contiene l'intersezione con il secondo estremo della traccia;
\item Nel secondo sotto-poligono se il punto di intersezione con il secondo estremo della traccia è già stato superato.
\end{enumerate}
Per verificare la condizione di intersezione ripetiamo per tutti i lati il sistema \eqref{eq:intersezioni}. Quindi se esiste l'intersezione e il parametro $t \in [0,1]$ allora il secondo estremo della traccia sta sul lato considerato e da quel punto in avanti dovremo memorizzare i dati nel secondo sotto-poligono. Il passaggio da un poligono all'altro può essere segnalato da un booleano che si attiva non appena troviamo l'intersezione. Contemporaneamente alla memorizzazione dei lati, allo stesso modo si procede con quella dei vertici. Per ultimo aggiungiamo ad entrambi i sotto-poligoni il lato che divide li divide.
\\ [2mm]
Con l'accortezza di analizzare tutti i poligoni adiacenti al lato di prima intersezione, questo procedimento viene iterato per ogni traccia del poligono originale finché il punto di intersezione della traccia con i lati del poligono, ottenuto mediante il sistema \eqref{eq:intersezioni}, ha l'ascissa curvilinea $s \in [0,1]$. Non appena troveremo un valore al di fuori di questo intervallo, concluderemo la divisione del poligono in cui è presente la traccia e passeremo alla traccia successiva, fino ad esaurirle.



\end{document}
